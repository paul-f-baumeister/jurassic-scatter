\section{Quick Start}

get code from repository

(this will create directory jurassic with the subdirectories: docu, examples-tab, lib, jurassic-1.0)

run install script (check environment variables, lib\/build.sh, make) (maybe one that works for JUROPA, JUQUEEN and normal workstations)

go to examples and run clear.sh

If this works you can try the other examples or start your own projects. 

If any error occours, please refer to the more detailed documentation.

\section{download the code}
jurassic repository

optional/additional file repository (tables and filter functions, refractive index files, external optical properties databases, Remedios atmospheres and PSC study???)

\section{Installation}
\begin{itemize}
\item which libraries are needed? GSL and NCDF
\item how to compile and link the code?
\end{itemize}

\section{Code Execution}
\begin{itemize}
\item how to run the code?
\item which input is required? (link to description section)
\item examples for forward simulations: clear air, aerosol scattering, retrieval
\end{itemize}

\subsection{Forward Simulations}
\begin{itemize}
\item \*.ctl: control file with global control flags
\item atm.tab: atmospheric profiles/mixing ratios
\item obs.tab: observations and geometry
\item rad.tab: forward simulation results output file name
\item tables
\item \emph{aero.tab}: aerosol and cloud parameters
\item \emph{complex refractive indices}
\item or \emph{optical properties database}
\end{itemize}

\begin{itemize}
\item maybe one example for clear air nadir
\item another example for cloudy limb
\end{itemize}

\subsubsection{Clear Air Simulation}
\begin{minted}
[
frame=lines,
bgcolor=peach,
formatcom=\color{black},
linenos
]
{bash}

# path
pirat=../pirat-1.0

info "Create geometry..."
$pirat/limb test.ctl 800 5 15 1 obs.tab || exit

info "Call forward model..."
$pirat/formod test-aerosol.ctl obs.tab atm.tab rad.tab || exit
\end{minted}

\subsubsection{Aerosol/Cloud Simulation}
\begin{minted}
[frame=lines,
bgcolor=peach,
formatcom=\color{black},
linenos]{bash}
#$pirat/formod aerosol1.ctl obs.tab atm.tab rad.tab || exit
#$pirat/formod aerosol1.ctl obs.tab atm.tab rad.tab DIRLIST dirlist-aero.asc|| exit
$pirat/formod aerosol1.ctl obs.tab atm.tab rad.tab AEROFILE aero.tab|| exit
#$pirat/formod aerosol1.ctl obs.tab atm.tab rad.tab DIRLIST dirlist-aero.asc AEROFILE aero.tab|| exit
#$pirat/formod test.ctl obs.tab atm.tab rad.tab TASK c || exit

\end{minted} 
%$
Instead of giving the filename of the aerosol data here it can also ge given in the control-file:

\texttt{AEROFILE = aero.tab}

The same holds for \texttt{DIRLIST} and \texttt{TASK}.

% \begin{minted}[bgcolor=black,formatcom=\color{white}]{bash}
% # path
% pirat=../pirat-1.0

% info "Create geometry..."
% $pirat/limb test.ctl 800 5 15 1 obs.tab || exit

% info "Call forward model..."
% $pirat/formod test-aerosol.ctl obs.tab atm.tab rad.tab || exit
% \end{minted}

\subsection{Retrieval}
\begin{itemize}
\item \*.ctl: control file with global control flags
\item atm.tab: atmospheric profiles/mixing ratios
\item obs.tab: observations and geometry
\item rad.tab: forward simulation results output file name ???
\item tables
\item measurement data converted to obs.tab
\item maybe one example for clear air nadir
\item another example for clear air limb
\end{itemize}